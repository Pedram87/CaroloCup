% Sample file on how to use subfiles.
\documentclass[Report.tex]{subfiles}

\begin{document}

\chapter{Discussion}
\label{chapter:Discussion}

This chapter gives a detailed discussion about the results showed in
chapter~\ref{chapter:Results} highlighting on lurking anomalies and providing
possible solutions for these issues.

As for the bird's eye view, its transformation is the most computation
demanding from all the image processing procedures and clustering algorithm.
Therefore, it is of utmost importance to find an alternative solution for this
transformation. To gain a more precise view of this issue, separate tests were
performed for the warpPerspective OpenCV function and for the calculation of
the transformation matrix M as well. Since, the latter possesses negligible
run-time property compared to the former function's property, an alternative,
less computational implementation is highly recommended for this
warpPerspective function. To further emphasize the significance of this matter,
it must be pointed out, as table~\ref{tab:result} shows, that more than half of
the computation is consumed only by this function, and this statement is
hardware independent.

Furthermore, the lines detected by the Canny edge-detection algorithm, as
figures \ref{fig:img_proc} show, are distorted due to the bird's eye view
transformation. This distortion can be explained by the order of the image
processing steps. Namely, the Canny edge-detection precedes the bird's eye view
transformation which is succeeded by the Hough line transformation. This
sequence allows the bird's eye view transformation to distort the detected
edges. However, the sequence of steps where the bird's eye view transformation
and Canny edge-detection are switched would lead to a distortion free bird's
eye view transformation. This modification is relevant for the detection of the
two solid lines.

Lastly, the current implementation of pattern matching is somewhat limited,
since it can only detect straight lines that are situated in a relatively
vertical position. This limitation does not affect the processing of straight
lines, since we can assume that the camera always points approximately the same
direction as the lane goes. Although this strategy is not applicable in curve
scenarios. For these cases a separate curve fitting algorithm should be applied
on the clusters to detected the solid and dashed curved lines' angles and
compute the optimal line.
\end{document}
