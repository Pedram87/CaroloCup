\documentclass[Report.tex]{subfiles}

\begin{document}

\chapter{Introduction}
This report provides solutions for lane following functionality of
self-driven miniature vehicles. The lane following contains the following parts:
\begin{itemize}
  \item image processing,
  \item clustering of lines,
  \item pattern matching on clusters,
  \item defining input data for the control algorithm.
\end{itemize}

\section{Image Processing} % (fold)
\label{sec:Image Processing}
Image processing involves the processing of the frames from the camera,
filtering noise on the frames, transforming the filtered frames to bird's
eye view, and performing a Hough line transformation on the bird's eye
view image to get the line shapes from the frame.
% section Image Processing (end)

\section{Clustering of lines} % (fold)
\label{sec:Clustering of lines}
A density-based spatial clustering algorithm was used to group the
neighbouring lines into clusters by their distance and minimum numbers of
lines. This algorithm is an alternated DBSCAN (density-based spatial clustering of
applications with noise)\cite{www:dbscan} algorithm.
The classical DBSCAN algorithm operates on a set points and returns a set of
clusters where each point is uniquely assigned to exactly one cluster. This
version of the algorithm does not care about the fact that these points were
connected once, and therefore should belong to the same cluster. The
algorithm used in this project, on the other hand, operates on a set of lines
and returns a set of clusters of points. This alternation of the algorithm
provides more accurate clustering for the purpose of this project and requires
less computation since we only iterate through half of the points.
% section Clustering of lines (end)

\section{Pattern Matching on Clusters} % (fold)
After having the clusters, there is only one relevant step remained, which is
finding the clusters that contain the one dashed line and the two solid lines
from the lane. This is done by finding the greatest difference on the x and y
axes between any pair of points in the cluster and checking if these values
fall within a predefined range.
% section Pattern Matching on Clusters (end)

\section{Input data for the control algorithm} % (fold)
The control algorithm operates on two lines, the dashed line and the solid line
on the right. Therefore, the clusters containing the dashed and solid lines
must be transformed into lines that can be fed to the control algorithm. This
transformation is done by checking the Euclidean distance between each pair of
points in the given cluster to find the one with the greatest value. The solid
line on the left is only used during overtaking.
% section Input data for control algorithm (end)

\section{Structure of the Report} % (fold)
This chapter gave a brief introduction to the topic and structure of this
report. The next chapter, chapter~\ref{chapter:Theory}, will give a thorough
description about the image processing, line clustering and pattern matching
methods. Chapter \ref{chapter:Results} shows the performance data measured on
the target hardware. Chapter \ref{chapter:Discussion} provides a discussion
covering deficiencies in the current implementation and design and propose
possible solutions for these deficiencies. Lastly, chapter
\ref{chapter:Conclusion} contains the final conclusions of this project and
summarizes its results.

\end{document}
